%%%%%%%%%%%%%%%%%%%%%%%%%%%%%%%%%%%%%%%%%%%%%%%%%%%%%%%%%%%%%%%%%%%%%%%%%%%%%
%
% This file is delivered as an usage example of the file 'kkr13-eng.sty'.
% The style file 'kkr13-eng.sty' has been prepared for papers intended 
% to be presented at the XIII National Conference on Robotics in Poland.
%
% The file 'kkr13-eng.sty' contains definitions of macros for English
% and Polish authors. Thus their names are in English and Polish respectively.
% They define the same operations. English and Polish names are used
% only for authors' convenience. 
% Therefore both types of macros can be used in this type of the document.
% The same type of macros are defined in the style file defined for papers 
% written in Polish.
% Below there is a list of macros names in English and their equivalent forms
% in Polish.
%
%       \articleTitle          =   \tytulArtykulu
%       \titleFootNote         =   \stopkaPrzypisTytulu
%       \articleAuthor         =   \autorArtykulu
%       \affiliation           =   \instytucja
%       \theSameAffiliationAs  =   \instytucjaTaSamaJak
%       \articleShortTitle     =   \naglowekTytulSkrocony
%       \breakAuthorsLine      =   \zlamLinieAutorow
%       \authorsForHeader      =   \naglowekAutorzyArtykulu
%       \authorsCustomizedList =   \wlasnaListaAutorow
%       \abstract              =   \streszczenie
%       \Equation              =   \wzor
%       \refeq                 =   \refwzor
%       \reffig                =   \refrys
%       \refFig                =   \refRys
%
% Author: Bogdan Kreczmer 
% Last modification:      2013.12.09
% 
% In the case of any doubt, please, contact via e-mail.
% The address of the author is:
%  bogdan.kreczmer@pwr.wroc.pl
%
%%%%%%%%%%%%%%%%%%%%%%%%%%%%%%%%%%%%%%%%%%%%%%%%%%%%%%%%%%%%%%%%%%%%%%%%%%%%

\documentclass[11pt,twoside]{article}
\usepackage{kkr13-eng}


\articleTitle{3D Object Recognition Based on RGBD Images}

%
% For information of a project (if any)
%
\titleFootNote{The work described in this paper was conducted with author's Bachelor's dissertation}

%
% Parameters:
%   1 - initials of the first and middle name of an author
%   2 - the first and middle name of an author
%   3 - the last name of an author
%
\articleAuthor{A.}{Adam}{Kosiorek}

\affiliation[institute1]{ Institute of Automation and Robotics, The Faculty of Mechatronics of Warsaw University of\\ Technology, email: iair@mchtr.pw.edu.pl, website: http://iair.mchtr.pw.edu.pl}


\begin{document}

\abstract{ %
 The paper presents a very short example of usage of the style file named
 {\tt kkr13-eng.sty}.
}

\maketitle

\section{ TEXT OF THE PAPER }

The instruction presents typographic layout of the text
and technical requirements 
 to the papers intendant to be presented at the Conference.
Fonts sizes are declared in points (1pt =  0,35146mm).


\subsection{For {\LaTeX} users}

 To fulfill the requirements described below a style file {\tt kkr13-eng.sty}
 can be used. It is available at the Conference website
 {\tt http://kkr13.pwr.wroc.pl}. The bibliography style coincides with 
 the plain BibTeX style {\tt plain.bst}.


\subsection{ The manuscript preparation}

 The total number of pages for a regular paper should not exceed 10 pages.
 The authors of the papers for plenary sessions are recommended not to
 exceed 20 pages (the number of pages should be even).
 The text of the paper has to be written using 
 the 11pt font size. The space between lines should be equal to 5mm, 
 the width of the text is 13.1cm, the height is 20.8cm. 
 The indentation is 1.25cm 
 (the indentation must not be created at the beginning of a section or 
 subsection). The alignment of the text is centered along horizontal line.
 The format of a page is A4. 



\subsubsection{\small Mathematical formulae}

 The number of a formula has to be written
 in parentheses shifted to the right (see below). Symbols of variables 
 should be written in italic.
 \begin{equation}
   x = \frac{a^{(1-z)} + b}{10 - d_2}
 \end{equation}

 \noindent
 Intervals of values should be written without spaces, e.g.\ 4--45MPa.


\section{FIGURES AND TABLES}

\subsection{Figures and their arrangement}

 Figures should be numbered with the Arabic numerals (see fig. 
 \ref{fig-example}). 

 \begin{figure}[hbtp]
  \fbox{\rule{0mm}{45mm}\rule{85mm}{0mm}}
  \caption{The figure caption}
  \label{fig-example}
 \end{figure}

\subsection{Tables}

 Tables have to be formated in the way presented below (see table 
 \ref{tab-example}). The tables have to be numbered  separately from 
 the figures.

 \begin{table}[hbtp]
  \caption{The table caption}
   \label{tab-example}
   \begin{tabular}{|c|c|c|c|c|}\hline
       Column 1 & Column 2 & Column 3 &  Column 4 & Column 5\\ \hline
       $\vdots$ & $\vdots$ & $\vdots$ & $\vdots$ & $\vdots$\\ \hline
    \end{tabular}
 \end{table}

\section{CITATIONS}

Citations of papers should be marked  with square brackets, e.g.
\cite{Ming}. The references have to be arranged in 
the alphabetic order relative to the authors' surnames (see the example
below).

\begin{thebibliography}{99}
\bibitem{Litwiniec}
    A. Litwin et al. Transport phenomena in InSb doped with various impurities.
    In: 11 International Conference on the Physics of Semiconductors.
   {\em Proceedings}. Warszawa -- Poland, July 25--29, 
    1972. Vol. 2, pp. 952--957.

 \bibitem{Lamadrid} 
  J.~G. de~Lamadrid.
  \newblock Avoidance of obstacles with unknown trajectories: Locally optimal
   paths and periodic sensor readings.
   \newblock {\em The Inter. Journal of Robotics Research}, 13(6):496--507,
   December 1994.

 \bibitem{Ming}
   A~Ming, K.~Kajihara, M.~Kajitani, and M.~Shimojo.
   \newblock Development of a rapid obstacle sensing system using sonar 
   ring for mobile robot.
   \newblock In {\em Proc. Int. Conf. on Robotics and Automation}, 
   volume~3, pp. 3068 -- 3073, Washington, D.C., May 2002.

 \bibitem{Singh:1987:Convex:Shapes}
   J.~Sanjiv Singh and Meghanad~D. Wagh.
   \newblock Robot path planning using intersecting convex shapes: Analysis and
   simulation.
   \newblock {\em IEEE Journal of Robotics and Automation}, RA-3(2):101 -- 108,
   April 1987.
\end{thebibliography}
\end{document}



