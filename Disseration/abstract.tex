\begin{center}
\begin{LARGE}\textbf{Streszczenie}\end{LARGE}
\end{center}

\vspace{1.0cm}

	W pracy przedstawiono podejście Bag of Words do klasyfikacji obiektów na podstawie trójwymiarowych chmur punktów. Wykorzystano metodę uczenia nadzorowanego, dzięki czemu można rozpoznawać dowolne kategorie obiektów. Opisano metodologię BoW, jej znaczenie w wizjii komputerowej, a także algorytmy znajdujące zastosowanie w poszczególnych stadiach procesu. Następnie przedstawiono problem klasyfikacji. Określono wymaganą funkcjonalność, zaprojektowano konfigurowalną i łatwo rozszerzalną aplikację oraz zaimplementowano ją w C++. Zbadano algorytmy pod kątem przydatności do BoW oraz sprawdzono skuteczność aplikacji na dwóch powszechnie uznawanych zbiorach danych.

\vspace{3.0cm}

\begin{center}
\begin{LARGE}\textbf{Abstract}\end{LARGE}
\end{center}

\vspace{1.0cm}

	This paper introduces a Bag of Words semantic object classification framework based on three dimensional point clouds. The framework uses machine learning algorithms and a learning by example approach, thus an arbitrary set of categories can be used. Basic concepts of BoW and its role in computer vision are introduced. Then, the most popular algorithms for different steps of BoW are described and a problem of classification is explained. After a requirement analysis was performed a configurable and extensible application was designed and implemented in C++. Finally, algorithms for different steps of BoW classification are evaluated and results on two benchmark datasets are discussed.
	

	
	


